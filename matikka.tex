\documentclass{article}
\usepackage{mathtools} % For advanced math formatting
\usepackage{amsmath}
\usepackage{amssymb}

\begin{document}
	
	\title{Matematiikan ja tilastotieteen osasto
		\newline Johdatus yliopistomatematiikkaan
		\newline  MAT11001 – Harjoitus 1 – Syksy 2024
		\newline Viimeinen palautuspäivä 11.9.2024}
	\author{Juha-Pekka Heikkilä}
	\date{\today}
	
	\maketitle
	
	
	\section{Tehtävä 1}

	
\begin{enumerate}


	\item[(a)]
	\[
	\pi \notin \mathbb{Q}
	\]	
	
	\item[(b)]
	\[
	\{-5, 1, 27, 100\} \subset \mathbb{Z}
	\]

	\item[(c)]
	\[
	C \neq \emptyset \quad \text{ja} \quad C \subset A \setminus B
	\]

	\item[(d)]
	\[
	\{ x \in \mathbb{R} \mid  \sqrt[3]{x} \geq -0,123 \ \text{ja} \  \sqrt[3]{x} \leq 1,234 \}
	\]

	\item[(e)]
	\[
	p \in \mathbb{Q} \ \text{ja} \ q \in \mathbb{Q} \implies p + q \in \mathbb{Q}
	\]

	\item[(f)]
	\[
	\{b\} \cup B \subset \complement A
	\]

	\item[(g)]
	\[
	n \in \{\mathbb{N} \mid n > 1 \mid \exists a, b \in \mathbb{N}, \ 1 < a < n, \ 1 < b < n, \ a \cdot b = n \} 
	\]

	\item[(h)]
	\[
	\{ x \in \mathbb{R} \mid x \in [1, 3]\} \cap \{ y \in \mathbb{R} \mid y\ in ]2, 4[\} = \{ z \in \mathbb{R} \mid z \in [2, 3[\}
	\]

	
\end{enumerate}






	\section{Tehtävä 2}

	\( $A = \{1, 2, 3, 4, 5, 6\}$ \ \text{ja} \ $B = \{2, 4, 6, 8\}$\)

\begin{enumerate}

	\item[(a)]
	\[
	A \cup B = \{1, 2, 3, 4, 5, 6, 8\}
	\]
	\[
	A \cap B = \{2, 4, 6\}
	\]

	\item[(b)]
	\[
	A \setminus B = \{ 1, 3, 5\}
	\]
	\[
	B \setminus A = \{ 8\}
	\]

\end{enumerate}





	
	Here is an inline formula: \(E = mc^2\).
	
	Below is a displayed formula:
	
	\[
	\int_{a}^{b} x^2 \, dx = \frac{b^3}{3} - \frac{a^3}{3}
	\]
	
	And here is an example with multiple equations:
	
	\begin{align}
		a^2 + b^2 &= c^2 \\
		\sin^2(x) + \cos^2(x) &= 1
	\end{align}


	This is test \(\frac{3}{6}\) \\
	This is test \(\frac{3}{6}\)
	
	\[
	\frac{3}{6} 
	\]
	
	
	
	\[
	\frac{\frac{1}{x}+\frac{1}{y}}{y-z}
	\]
	
\end{document}