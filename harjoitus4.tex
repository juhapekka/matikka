\documentclass{article}
\usepackage{mathtools}
\usepackage{amsmath}
\usepackage{amssymb}
\usepackage{tikz}
\usepackage{lipsum}
\usepackage{graphicx}
\usepackage[finnish]{babel}
\usepackage{fancyhdr}

\pagestyle{fancy}

\newcounter{tehtava}
\setcounter{tehtava}{1}


\fancyhf{}
\fancyhead[L]{Tehtävä \thetehtava}

\fancypagestyle{plain}{
    \fancyhf{}
    \fancyhead[L]{Tehtävä \thetehtava}
}

\begin{document}
    \thispagestyle{plain}
	\title{MAT11001 - Harjoitus 4}
	\date{}
	\maketitle
	
	
	\section*{Tehtävä \thetehtava}
    Olkoon $f : X \rightarrow Y$. Osoita, että kaikilla $C, D \subset Y$ pätee\newline
    $f^{-1} [C \cup D] = f^{-1}C \cup f^{-1}D$
    
    \paragraph*{Osoitetaan $f^{-1} [C \cup D] \subseteq f^{-1}C \cup f^{-1}D$}\quad\newline
    Olkoon $x \in f^{-1} [C \cup D]$\newline
    \begin{itemize}
        \item Jos $f(x) \in C \cup D$, niin $f(x) \in C$ tai $f(x) \in D$ (yhdistelmä määritelmän mukaan)
        \begin{itemize}
            \item Jos $f(x) \in C$, niin $x \in f^{-1}(C)$
            \item Jos $f(x) \in D$, niin $x \in f^{-1}(D)$  
        \end{itemize}
        \item Siis $x \in f^{-1}(C)$ tai $x \in f^{-1}(D)$, eli $x \in f^{-1}(C) \cup f^{-1}(D)$
    \end{itemize}
    Johtopäätös: $f^{-1} [C \cup D] \subseteq f^{-1}(C) \cup f^{-1}(D)$

    \paragraph*{Osoitetaan $f^{-1}(C) \cup f^{-1}(D) \subseteq f^{-1} [C \cup D] $}\quad\newline
    Olkoon $x \in f^{-1}(C) \cup f^{-1}(D)$
    \begin{itemize}
        \item Tämä tarkoittaa, että $x \in f^{-1}(C)$ tai $x \in f^{-1}(D)$
        \item Jos $x \in f^{-1}(C)$, niin $f(x) \in C$
        \item Jos $x \in f^{-1}(D)$, niin $f(x) \in D$
        \item Koska $f(x) \in C$ tai $f(x) \in D$, pätee $f(x) \in C \cup D$
        \item Siis $x \in f^{-1} [C \cup D]$
    \end{itemize}
    \textbf{Eli $f^{-1}[C \cup D] = f^{-1}C \cup f^{-1}D$}
    \paragraph*{Kokeillaan $f : \mathbb{R} \to \mathbb{R}$, $f(x) = |x|$, ja joukot $C = [0, 2]$ ja $D = [1, 3]$}\quad\newline
    \textbf{Tavoite:} Määritetään ensin joukko $f^{-1}[C \cup D]$ ja sitten joukko $f^{-1}C \cup f^{-1}D$
    \begin{itemize}
        \item[a)] $C \cup D = [0, 2] \cup [1, 3] = [0, 3]$
        \item[b)] Määritetään alkukuva $f^{-1}[C \cup D]$
        \begin{itemize}
            \item $f(x) = |x|$ kuvaa jokaisen $x \in \mathbb{R}$ itseisarvoonsa
            \item $|x| \leq 3$, eli $-3 \leq x \leq 3$
            \item siksi $f^{-1}[C \cup D] = [-3, 3]$
        \end{itemize}
    \end{itemize}
    \quad\newline
    \textbf{Määritä $f^{-1}C \cup f^{-1}D$}
    \begin{itemize}
        \item $f^{-1}C$:
        \begin{itemize}
            \item $C = [0, 2] \implies 0 \leq |x| \leq 2$
            \item $f(x) = |x|$ saa arvot $0 \leq |x| \leq 2 \ x$-arvoilla: $f^{-1}(C) = [-2, 2]$
        \end{itemize} 
        \item $f^{-1}D$:
        \begin{itemize}
            \item $D = [1, 3] \implies 1 \leq |x| \leq 3$
            \item $f(x) = |x|$ saa arvot $1 \leq |x| \leq 3 \ x$-arvoilla $f^{-1}(D) = [-3, -1] \cup [1, 3]$
        \end{itemize} 
        \item yhdistetään tulokset: $f^{-1}(C) = [-2, 2]$ ja $f^{-1}(D) = [-3, -1] \cup [1, 3] \implies 
        f^{-1}(C) \cup f^{-1}(D) = [-3, 3]$
    \end{itemize}
    \quad\newline
    Siis: $f^{-1}[C \cup D] = [-3, 3]$ ja $f^{-1}(C) \cup f^{-1}(D) = [-3, 3]$

    \newpage
    \stepcounter{tehtava}
	\section*{Tehtävä \thetehtava}
    Olkoot $X = \{1, 2, 3\}$ ja $Y = \{1, 2\}$. Keksi kuvaus $X \rightarrow  Y$ tai $Y \rightarrow X$ tai $X \rightarrow X$ tai $Y \rightarrow Y$, joka
    \begin{itemize}
        \item[a)] on injektio, mutta ei ole surjektio\newline
        $Y \rightarrow X$ \newline
            \begin{tikzpicture}
                % Nodes for set Y
                \node (Y1) at (0,2) {1};
                \node (Y2) at (0,1) {2};
            
                % Nodes for set X
                \node (X1) at (3,2) {1};
                \node (X2) at (3,1) {2};
                \node (X3) at (3,0) {3};
            
                % Arrows
                \draw[->] (Y1) -- (X1);
                \draw[->] (Y2) -- (X2);
            \end{tikzpicture}
            \item[b)] on surjektio, mutta ei ole injektio,\newline
        $X \rightarrow Y$ \newline
            \begin{tikzpicture}
                % Nodes for set Y
                \node (Y1) at (0,3) {1};
                \node (Y2) at (0,2) {2};
                \node (Y3) at (0,1) {3};
            
                % Nodes for set X
                \node (X1) at (3,3) {1};
                \node (X2) at (3,2) {2};
            
                % Arrows
                \draw[->] (Y1) -- (X1);
                \draw[->] (Y2) -- (X2);
                \draw[->] (Y3) -- (X2);
            \end{tikzpicture}
            \item[c)] ei ole surjektio eikä injektio,\newline
        $X \rightarrow Y$ \newline
            \begin{tikzpicture}
                % Nodes for set Y
                \node (Y1) at (0,3) {1};
                \node (Y2) at (0,2) {2};
                \node (Y3) at (0,1) {3};

                % Nodes for set X
                \node (X1) at (3,3) {1};
                \node (X2) at (3,2) {2};

                % Arrows
                \draw[->] (Y1) -- (X1);
                \draw[->] (Y2) -- (X1);
            \end{tikzpicture}
            \item[d)] on bijektio,\newline
        $Y \rightarrow Y$ \newline
            \begin{tikzpicture}
                % Nodes for set Y
                \node (Y1) at (0,2) {1};
                \node (Y2) at (0,1) {2};
            
                % Nodes for set X
                \node (X1) at (3,2) {1};
                \node (X2) at (3,1) {2};
            
                % Arrows
                \draw[->] (Y1) -- (X1);
                \draw[->] (Y2) -- (X2);
            \end{tikzpicture}
        \end{itemize}

        \newpage
        \stepcounter{tehtava}
        \section*{Tehtävä \thetehtava}
    
\end{document}