\documentclass{article}
\usepackage{mathtools}
\usepackage{amsmath}
\usepackage{amssymb}
\usepackage[finnish]{babel}

\begin{document}
	
	\title{MAT11001 – Harjoitus 2}
	\date{}
	\maketitle
	
	
	\section*{Tehtävä 1}
    \(A\) = \{1, 2, 3, 4\} ja \(B\) = \{1, 2, 3\} \\[10pt]
    Määritä \(A\) × \(B\) ja \(B\) × \(A\).
    Onko \(A\) × \(B\) = \(B\) × \(A\)?\\
	\[
        \begin{aligned}
            A \times B &= \{(a,b) \mid a \in A \ \text{ja} \ b \in B\} \\
            &= \{(1,1), (1,2), (1,3), (2,1), (2,2),(2,3),(3,1), (3,2),(3,3), (4,1), (4,2),(4,3)\} \\[10pt]
            B \times A &= \{(b,a) \mid b \in B \ \text{ja} \ a \in A\} \\
            &= \{(1,1), (1,2), (1,3), (1,4), (2,1),(2,2),(2,3), (2,4),(3,1), (3,2), (3,3),(3,4)\}  \\[10pt]
            &\implies A \times B \neq  B \times A
        \end{aligned}
    \] 
    \newpage
	\section*{Tehtävä 2}
    \[
        \begin{aligned}
            (A \cup B) \times C &= (A \times C) \cup (B \times C) \\[10pt]
	        &(x,y) \in (A \cup B) \times C \\
            &\implies x \in A \ tai \ x \in B \text{ ja } y  \in C\\[10pt]
            &\text{jos } x \in A \text{ niin } (x,y) \in A \times C\\
            &\text{jos } x \in B \text{ niin } (x,y) \in B \times C\\[10pt]
            &\implies (x,y) \in (A \times C) \cup (B  \times C)\\[10pt]
            \text{Tämä osoittaa että }(A \cup B) \times C &= (A \times C) \cup (B \times C) \\[10pt]
            A = \{1, 2, 3\}, B = \{2, 3, 4\} \text{ ja } C &= \{1, 2\} \\
            A \cup B &= \{ 1, 2, 3, 4 \} \\
            (A \cup B) \times C &= \{(1,1),(1,2),(2,1),(2,2),(3,1),(3,2),(4,1),(4,2) \} \\[10pt]
            A \times C &= \{(1,1), (1,2),(2,1), (2,2), (3,1),(3,2)\} \\
            B \times C &= \{(2,1), (2,2),(3,1), (3,2), (4,1),(4,2)\} \\
            (A \times C) \cup (B \times C) &= \{(1,1), (1,2),(2,1), (2,2), (3,1),(3,2),(4,1), (4,2) \} \\[10pt]
            \implies &\text{Molemmat joukot ovat samat }(A \cup B) \times C = (A \times C) \cup (B \times C)
        \end{aligned}
    \]
    \newpage
	\section*{Tehtävä 3}
    \[
        \begin{aligned}
            \text{Osoita, että jos n} \in \mathbb{N} \text{, niin n}^2 \text{ + n + 1 on pariton.} \\[5pt]
            \text{Jos n on parillinen, niin } \\
            n = 2k \text{ missä } k \in \mathbb{N} \implies n^2 + n + 1 = (2k)^2 + 2k + 1 \\
            \text{Huomataan että } 4k^2 + 2k \text{ on parillinen, joten } 4k^2 + 2k + 1 \text{ on pariton.} \\[10pt]
            \text{Jos n on pariton, niin } \\
            n = 2k + 1 \text{ missä } k \in \mathbb{N} \implies  n^2 + n + 1 = (2k + 1)^2 + (2k + 1) + 1 \\
            \implies (2k + 1)^2 = 4k^2 + 4k + 1 \\
            \implies 4k^2 + 4k + 1 + 2k + 1 + 1 = 4k^2 + 6k + 3 \\
            \text{Huomataan että } 4k^2 + 6k \text{ on parillinen, joten } 4k^2 + 6k + 3 \text{ on pariton.} \\[20pt]
            \text{Molemmissa tapauksissa, n on parillinen tai pariton, } n^2 + n + 1 \text{ on pariton } \\
            \text{Joten kun } n \in \mathbb{N} \text{ niin } n^2 + n + 1 \text{ niin on pariton.}
        \end{aligned}
    \]
    \newpage
	\section*{Tehtävä 4}
    \[
        \begin{aligned}
            &\text{Osoita että 3 + 11 + ... + (8n - 5) = 4n}^2 \text{ - n kaikilla n } \in \mathbb{N}, n \geq 1 \\[15pt]
            &\text{kun n = 1 :} \\
            &\implies 8 * 1 - 5 = 4 * 1^2 - 1 = 3\\[15pt]
            &\text{Oletetaan n = k} \\
            &\implies 3 + 11 + ... + (8k - 5) = 4k^2 - k\\[15pt]
            &\text{Todistetaan että väite pätee kun n = k + 1} \\
            &3 + 11 + ... + (8k - 5) + [8(k + 1) - 5] = 4(k + 1)^2 - (k + 1)\\[15pt]
            &\text{Vasen puoli:} \\
            &S(k + 1) = S(k) + [8(k + 1) - 5]\\
            &= 4(k^2 - k) + [8k + 8 - 5]\\
            &= 4k^2 - 7k + 3\\[15pt]
            &\text{Oikea puoli:} \\
            &4(k + 1)^2 - (k + 1) = 4(k^2 +2k + 1)  - (k + 1) = 4k^2 + 8k + 4 - k - 1\\
            &= 4k^2 + 7k + 3\\[15pt]
            &\text{Koska perus askel ja induktio askel pätee voimme päätellä että se pätee kaikilla $n \in \mathbb{N}$, $n \geq 1$.} \\
        \end{aligned}
    \]

\end{document}