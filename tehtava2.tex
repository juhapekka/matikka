\documentclass{article}
\usepackage{mathtools}
\usepackage{amsmath}
\usepackage{amssymb}
\usepackage[finnish]{babel}

\begin{document}
	
	\title{MAT11001 – Harjoitus 2}
	\date{}
	\maketitle
	
	
	\section*{Tehtävä 1}
    \(A\) = \{1, 2, 3, 4\} ja \(B\) = \{1, 2, 3\} \\[10pt]
    Määritä \(A\) × \(B\) ja \(B\) × \(A\).
    Onko \(A\) × \(B\) = \(B\) × \(A\)?\\
	\[
        \begin{aligned}
            A \times B &= \{(a,b) \mid a \in A \ \text{ja} \ b \in B\} \\
            &= \{(1,1), (1,2), (1,3), (2,1), (2,2),(2,3),(3,1), (3,2),(3,3), (4,1), (4,2),(4,3)\} \\[10pt]
            B \times A &= \{(b,a) \mid b \in B \ \text{ja} \ a \in A\} \\
            &= \{(1,1), (1,2), (1,3), (1,4), (2,1),(2,2),(2,3), (2,4),(3,1), (3,2), (3,3),(3,4)\}  \\[10pt]
            &\implies A \times B \neq  B \times A
        \end{aligned}
    \] 
    \newpage
	\section*{Tehtävä 2}
    \[
        \begin{aligned}
            (A \cup B) \times C &= (A \times C) \cup (B \times C) \\[10pt]
	        &(x,y) \in (A \cup B) \times C \\
            &\implies x \in A \ tai \ x \in B \text{ ja } y  \in C\\[10pt]
            &\text{jos } x \in A \text{ niin } (x,y) \in A \times C\\
            &\text{jos } x \in B \text{ niin } (x,y) \in B \times C\\[10pt]
            &\implies (x,y) \in (A \times C) \cup (B  \times C)\\[10pt]
            \text{Tämä osoittaa että }(A \cup B) \times C &= (A \times C) \cup (B \times C) \\[10pt]
            A = \{1, 2, 3\}, B = \{2, 3, 4\} \text{ ja } C &= \{1, 2\} \\
            A \cup B &= \{ 1, 2, 3, 4 \} \\
            (A \cup B) \times C &= \{(1,1),(1,2),(2,1),(2,2),(3,1),(3,2),(4,1),(4,2) \} \\[10pt]
            A \times C &= \{(1,1), (1,2),(2,1), (2,2), (3,1),(3,2)\} \\
            B \times C &= \{(2,1), (2,2),(3,1), (3,2), (4,1),(4,2)\} \\
            (A \times C) \cup (B \times C) &= \{(1,1), (1,2),(2,1), (2,2), (3,1),(3,2),(4,1), (4,2) \} \\[10pt]
            \implies &\text{Molemmat joukot ovat samat }(A \cup B) \times C = (A \times C) \cup (B \times C)
        \end{aligned}
    \]
    \newpage
	\section*{Tehtävä 3}
    \[
        \begin{aligned}
            \text{Osoita, että jos n} \in \mathbb{N} \text{, niin n}^2 \text{ + n + 1 on pariton.} \\[5pt]
            \text{Jos n on parillinen, niin } \\
            n = 2k \text{ missä } k \in \mathbb{N} \implies n^2 + n + 1 = (2k)^2 + 2k + 1 \\
            \text{Huomataan että } 4k^2 + 2k \text{ on parillinen, joten } 4k^2 + 2k + 1 \text{ on pariton.} \\[10pt]
            \text{Jos n on pariton, niin } \\
            n = 2k + 1 \text{ missä } k \in \mathbb{N} \implies  n^2 + n + 1 = (2k + 1)^2 + (2k + 1) + 1 \\
            \implies (2k + 1)^2 = 4k^2 + 4k + 1 \\
            \implies 4k^2 + 4k + 1 + 2k + 1 + 1 = 4k^2 + 6k + 3 \\
            \text{Huomataan että } 4k^2 + 6k \text{ on parillinen, joten } 4k^2 + 6k + 3 \text{ on pariton.} \\[20pt]
            \text{Molemmissa tapauksissa, n on parillinen tai pariton, } n^2 + n + 1 \text{ on pariton } \\
            \text{Joten kun } n \in \mathbb{N} \text{ niin } n^2 + n + 1 \text{ niin on pariton.}
        \end{aligned}
    \]
    \newpage
	\section*{Tehtävä 4}
    \[
        \begin{aligned}
            &\text{Osoita että 3 + 11 + ... + (8n - 5) = 4n}^2 \text{ - n kaikilla n } \in \mathbb{N}, n \geq 1 \\[15pt]
            &\text{kun n = 1 :} \\
            &\implies 8 * 1 - 5 = 4 * 1^2 - 1 = 3\\[15pt]
            &\text{Oletetaan n = k} \\
            &\implies 3 + 11 + ... + (8k - 5) = 4k^2 - k\\[15pt]
            &\text{Todistetaan että väite pätee kun n = k + 1} \\
            &3 + 11 + ... + (8k - 5) + [8(k + 1) - 5] = 4(k + 1)^2 - (k + 1)\\[15pt]
            &\text{Vasen puoli:} \\
            &S(k + 1) = S(k) + [8(k + 1) - 5]\\
            &= 4(k^2 - k) + [8k + 8 - 5]\\
            &= 4k^2 - 7k + 3\\[15pt]
            &\text{Oikea puoli:} \\
            &4(k + 1)^2 - (k + 1) = 4(k^2 +2k + 1)  - (k + 1) = 4k^2 + 8k + 4 - k - 1\\
            &= 4k^2 + 7k + 3\\[15pt]
            &\text{Koska perus askel ja induktio askel pätee voimme päätellä että se pätee kaikilla $n \in \mathbb{N}$, $n \geq 1$.} \\
        \end{aligned}
    \]
    \newpage
	\section*{Tehtävä 5}
    Määritä luvut $n \in \mathbb{N}$, joilla pätee $2^n \leq n^2$. Todista väitteesi.
    \newline
    \newline
    Kokeillaan pieniä $n$ arvoja:
    \[
        \begin{aligned}
            &n = 1 \implies 2 \leq 1 \text{ Ei päde}\\
            &n = 2 \implies 4 \leq 4 \text{ Pätee}\\
            &n = 3 \implies 8 \leq 9 \text{ Pätee}\\
            &n = 4 \implies 16 \leq 16 \text{ Pätee}\\
            &n = 5 \implies 32 \leq 25 \text{ Ei päde}\\
        \end{aligned}
    \]
    Osoitetaan induktiolla $n \geq 5$ pätee $2^n > n^2  $:
    \[
        \begin{aligned}
            \text{perus tapaus:}&\\
            &n = 5\\
            &2^5 = 32\\
            &5^2 = 25\\
            &32 > 25 \text{ (väite pätee)}\\[10pt]
            \text{Osoitetaan että kaikilla $n = k \geq 5$:}&\\
            &2^k > k^2\\
            \text{Induktioväite: Todistetaan, } & \text{että väite pätee myös n = k + 1}\\
            &2^{k+1} > (k+1)^2\\
            &\implies 2 * 2^k > (k + 1)^2 \mid \text{Käytetään induktion oletusta } 2^k > k^2\\
            &\implies 2 * k^2 > k^2 + 2k + 1 \mid \text{Vähennetään molemmilta puolilta } k^2\\
            &k^2 > 2k + 1\\
            &k^2 - 2k - 1 > 0 \mid \text{saadaan positiivnen juuri } k = 1 + \sqrt{2} \approx 2.41\\
            &\implies k^2 - 2k - 1 > 0 \text{ pätee kaikilla }k \geq 5 
        \end{aligned}
    \]
    Luonnolliset luvut joilla  $2^n \leq n^2$ pätee ovat:
    \newline
    $n = 2, 3, 4$

    \newpage
	\section*{Tehtävä 6}
    Määritellään lukujono $(z_{n})$ rekursiivisesti \newline
    $z_0 = 1, z_1 = 2, z_{n+1} = z_n + 2z_{n-1}, n \geq 1$ \newline

    \begin{enumerate}
        \item[(a)]
        lukujonon 8 ensimmäistä lukua
        \[
        \begin{aligned}
            z_0 &= 1\\
            z_1 &= 2\\
            z_2 &= 2 + 2 * 1 = 4\\
            z_3 &= 4 + 2 * 2 = 8\\
            z_4 &= 8 + 2 * 4 = 16\\
            z_5 &= 16 + 2 * 8 = 32\\
            z_6 &= 32 + 2 * 16 = 64\\
            z_7 &= 64 + 2 * 32 = 128\\
        \end{aligned}
        \]
        \item[(b)]
        Esitä $z_n$ pelkän $n$:n avulla eli ratkaise rekursiivinen yhtälö kohdan (a) perusteella päättelemällä.
        \[
        z_n = 2^n
        \]
        \item[(c)]
        Todista kohdan (b) tulos oikeaksi.\newline Todistetaan induktiolla: 
        \[
            \begin{aligned}
                \text{perustapaus:} n = 0& \\
                z_0 &= 1 \text{ ja } 2^0 = 1 \implies z_0 = 2^0\\
                n = 1& \\
                z_1 &= 2 \text{ ja } 2^1 = 2 \implies z_1 = 2^1\\[10pt]
            \text{Induktio oletus: } n = k& - 1 \text{ ja } n = k \text{ missä } k \geq 1\\
            \text{Induktio väite: } n = k& + 1 \\
            z_{k + 1} =& \ 2^{k + 1}\\
            \text{todistus:}\\
            z_{k + 1} =& \ z_k + 2z_{k - 1}\\
            =& \ 2^k + 2 \times 2^{k - 1}\\
            \implies & 2 \times 2^{k - 1} = 2^1 \times 2^{k - 1} = 2^{1 + k - 1} = 2^k\\
            \implies & 2 \times 2^{k - 1} = 2^k\\
            \implies & 2^{k + 1} = 2^k + 2^k = 2 \times 2^k\\[5pt]
            \text{joten:}&\\
            & 2 \times 2^k = 2^1 \times 2^k = 2^{k + 1}\\
            \implies & \ z_{k + 1} = 2^{k + 1}
            \end{aligned}
        \]
        

    \end{enumerate}

\end{document}