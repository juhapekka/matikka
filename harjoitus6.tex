\documentclass{article}
\usepackage{mathtools}
\usepackage{amsmath}
\usepackage{amssymb}
\usepackage{tikz}
\usepackage{lipsum}
\usepackage{graphicx}
\usepackage[finnish]{babel}
\usepackage{fancyhdr}
\usepackage{pgf,tikz}


\pagestyle{fancy}

\newcounter{tehtava}
\setcounter{tehtava}{1}


\fancyhf{}
\fancyhead[L]{Tehtävä \thetehtava}

\fancypagestyle{plain}{
    \fancyhf{}
    \fancyhead[L]{Tehtävä \thetehtava}
}

\begin{document}
    \thispagestyle{plain}
	\title{MAT11001 - Harjoitus 6}
	\date{}
	\maketitle
	
	
	\section*{Tehtävä \thetehtava}
    Jatkoa harjoituksen 5 tehtävään 5. Tarkastellaan siis kompleksilukuja $z = (3, -4)$ ja $w =(-1, 2)$
    \begin{itemize}
        \item[\textbf{a)}] Laske kompleksilukujen $z$ ja $w$ tulo
        \[
        \begin{aligned}    
            (3-4i) \cdot (-1+2i)& \\
            &= 3*(-1) + 3*2i -4i*(-1) -4i*2i \\
            &= -3 + 10i -8i^2 \quad \mid i^2 = -1 \\
            &= 5 + 10i
        \end{aligned}
        \]
        \item[\textbf{b)}] Mikä on kompleksiluvun $w$ käänteisluku? \newline
        Käytetään kaavaa
        \[
            \frac{1}{w} = \frac{1}{a + bi} = \frac{a - bi}{(a + bi)(a - bi)} = \frac{a - bi}{a^2 + b^2}
                \quad \mid
                \begin{aligned}
                &\text{Tämä johtuu siitä, että } \\ &(a + bi)(a - bi) = a^2 + b^2
                \text{on reaaliluku.}
                \end{aligned}
        \]
        Liittoluku $\overline{w} = -1 - 2i$\newline
        $\implies (w)(\overline{w}) = (-1 + 2i)(-1 - 2i) = (-1)^2 - (2i)^2 = 1 - (-4) = 5$\newline
        Tulokseksi saadaan 5 joka on reaaliarvo, jolloin $w$ käänteisluku saadaan:\newline
        \[
        \frac{1}{w} = \frac{\overline{w}}{(w)(\overline{w})} = \frac{-1 - 2i}{5} = \frac{-1}{5} - \frac{2i}{5}
        \]
        eli kompleksiluvun $w = -1 + 2i$ käänteisluku on:
        \[
            \frac{1}{w} = -\frac{1}{5} - \frac{2i}{5}
        \]
\pagebreak
        \item[\textbf{c)}] Laske kompleksilukujen $z$ ja $w$ osamäärä.\newline
        \[
            \frac{z}{w} = \frac{(z)(\overline{w})}{(w)(\overline{w})}
        \]
        Edellisestä tehtävästä näimme että $(w)(\overline{w}) = 5$ jolloin
        \[
        \begin{aligned}
            & \frac{z}{w} = \frac{(z)(\overline{w})}{5} \\
            \implies & \frac{z}{w} = \frac{(3-4i)(-1-2i)}{5} \\
            = & \frac{(-3 -6i +4i + 8i^2)}{5}
            = & \frac{-11 -2i}{5}
        \end{aligned}
        \]


        
    \end{itemize}


\newpage
\stepcounter{tehtava}
\section*{Tehtävä \thetehtava}
\end{document}